\citet{GorgeonSwanson09} studied one single article (Web 2.0), an
article edited by more than 1,000 different people, and showed that
the publication followed an S-curve pattern. Taking an actor network
theory perspective \citep{Latour05,AkrichCallonLatour06}, \citet{Swarts09}
showed how the building of an article, especially a polemic article
(clean coal), is a process of accumulation of facts, and of ''translation''
of arguments or proposals in facts, this accumulation of facts being
harder and harder to contest and thus to delete. It is partially confirmed
by \citet{Luytetal08}, who showed that ''a sizable number of error
edits occurs in the very first edit'' and if ''many more error edits
appear in the last third {[}part of the life of the article{]}, a
fifth of the errors {[}remaining{]} are attributable to the first
error edit.'' As early as 2004, \citet{ViegasWattenbergDave04} also
found that, in the English Wikipedia, there is a strong dependence
on the first edit for the global structure of the article. Finally,
\citet{Halfakeretal09} showed that ''the number of reviews a word
survives is a strong predictor of whether the edit that removes the
word will be reverted'' (p. 9). The question these articles raise
is how the negotiation around the facts is done, and what the structure
of the team needed to do it is, things we are going to analyze in
the next sub-section. \\

% FA articles.

Another important part of the literature looks at the characteristics
of the articles, and particularly of the articles recognized by the
producers as good (the Featured Articles, or FA). \citet{Luytetal08}
proposed a still actually categorization of the different algorithms
used to automatically retrieve the Feature Articles.

For instance, \citet{Lih04,Brandle05,WilkinsonHuberman07}, confirmed
by \citet{Ortega09}, found that after taking into account age and
visibility (using Pagerank as a proxy), those Featured Articles have
statistically more edits and editors. \citet{WohnerPeters09} refined
these analyses, showing that these articles are ''in general more
persistently edited than low quality articles and that on the other
side they have a stage of a high editing intensity in their lifecycles''.
(p. 7) 

\citet{Adleretal08b}, on the English Wikipedia, noted that ''incorporating
the authority of reviewers gives good and robust performance'' to
characterize FA articles. This authority is measured as evaluating
author's contributions life-length, based on the argument, brought
by \citet{Cross06}, that their persistence is a proof a quality (something
which is quite fragile, \citealp{Luytetal08} showed). \citet{Huetal07}
on a subset of articles, found the same result. \citet{Mcguinnessetal06},
based on the same idea as Pagerank, looked at the internal links pointing
to the articles (they named the ''trust ratio'') showed that Feature
Article are significantly more cited. 

Also, and coherent with the analyses done on the contributors, FA
articles have experienced editors participating to their redaction
\citep{SteinHess07}. More precisely, a fine tune of experimented
editors and fresh newcomers increases the likelihood for an article
to become FA \citep{RansbothamKane11}, which, in other contexts,
have been proven to be very important for group's creativity and efficiency
\citep[for instance, ][]{UzziSpiro05,Uzzi08}. \citet{WilkinsonHuberman07}
also found that these articles have more discussions on their talk
page, which is rather normal as, as pointed by \citet{ArazyNov10},
''processes for becoming a featured article explicitly require additional
coordination activities'' (p. 234).

But it seems that, in a first approach, the best indicators of an
Feature Article, at least for the English version \citep{Dalipetal09}
are the length and basic quality of the writing, as it is for open
source contribution, actually\footnote{\citet{HofmannRiehle09} found that for open-source, simple heuristics
are superior to the more complex text-analysis-based algorithms to
estimate the size and the importance of a commit in open-source projects.}: textual features related to length \citep[result already stressed by ][]{Blumenstock08},
structure and style (\citet{LipkaStein10} even obtained a better
result on FA identification with a machine learning approach on article
styles than \citeauthor{Blumenstock08}'s algorithm on length); and
those which count for the less, are the most complex features, such
as those based on link analysis.

This kind of analysis can also be done at portal, or subject level,
as did \citet{Poderi09}, with rather against-intuitive results, as
it seems from this analysis that subjects having more feature articles
(high density subjects in his terminology) have longer articles, but
less edit and contributors than low density subjects, while the ratio
between major and minor edits is the same in the two groups. It seems
also that there is more often a single major editor in the high density
subject articles. However, as stressed by the author, this study has
been done on a small subset of articles and should be extended to
confirm its results. \citet{Jones08} proposed an analysis of the
revision patterns of the articles applying for FA label, and showed
that the final structure of the article is very dependent on the first
one, editors tending to expend sentences, paragraphs and sections.
His study having been done on a small subset of the articles also
(10), he questioned the fact this unique revision pattern is global
or not, which seems to be the case, according to the studies done
on the whole set of articles.

\citet{Ibaetal10} proposed an explanation to these seemingly contradictory
results. According to them, there would be two types of FA (in the
English Wikipedia): (1) articles of narrow focus created by few subject
experts, and (2) articles about a broad topic created by thousands
of interested incidental editors. Considering the preferential attachment
mechanism, this phenomenon is self-maintaining as the more exposed
articles are, the more probable is the fact that people contribute
to them (a result studied by \citealp{RansbothamKaneLurie12}).
