
\paragraph{People and articles.}

\citet{Halfakeretal09} found ''strong evidence of ownership behaviors
in practice'' in the production of an article, especially in the
articles designed as ''Maintained'', according to \citet{Thom-SantelliCosleyGay09},
despite the fact that ownership of content is discouraged. And, even
if less geographically situated than Flickr contributors \citep{HechtGergle10b},
Wikipedia ones can often be associated with ''relatively small geographic
regions, usually corresponding to where those users were born or where
they presently live. Also, for many users, the geographic coordinates
of pages to which they contribute are tightly clustered'' \citep{LiebermanLin09}.
Finally, \citet{HardyFrewGoodchild12} geolocalized IP addresses for
the anonymous contributions to geotag articles in 21 language projects
and concluded that ''the likelihood of an anonymous contribution
to a geotagged Wikipedia article exponentially decreases as the distance
between the contributor and article locations increases''. All these
behaviors are coherent with the findings of \citet{Zhangetal10} who
concluded that it seems that people involve themselves on a very specific
themes, at article level, rather, than, for, instance at domain level,
after looking at a subset of articles on terrorism and comparing them
to the Terrorism Knowledge Base. As pointed by \citet{Thom-SantelliCosleyGay09},
this focus, these ownership behaviors, are not bad per se, especially
in the first stages of an article where a small team seems to be more
efficient. But it can lead to overprotection and can decrease the
final quality of an article (see bellow the discussion on the form
of the team).

This does not mean either that there is no coordination at project
level or that people can not be asked to joint a particular project.
On the contrary, \citet{Zhuetal11} showed how the personal pages
are specifically used to do so. But this must be fine tuned: \citet{Choietal10},
also analyzing the messages on the personal pages, showed that welcome
messages, assistance written in newcomers' pages are quite effective
to improve their contribution, when ''invitations led to steeper
declines in edits.'' Other actions can also attract contributors
and structure the teams or, more precisely the discussions, such as
template message on the articles: \citet{Rossietal10} studied the
role played by NPoV (neutral Policy Violation) templates, which if
not made completely explicit by their analysis, seem to be used to
settle evidence of a latent conflict \citep{DenBestenetal10}, and
thus to attract the attention of the community on a problem which
has to be solved, whereas other template messages seem to be more
treated as simple messages.

These recruiting actions play a crucial role in the construction of
a ''good team'' for writing an article, which appears to be a congregation
of experienced people having already work together with new talents.

\paragraph{Form of the team.}

\citet{KitturKraut08} showed that explicit coordination (talk) is
more efficient when there are few editors, when implicit coordination
(few editors editors concentrate the main part of the edits when the
majority is peripheral editor) is more efficient when there are more
editors. They also found that explicit coordination is needed more
at the early stage of the article. In any case, there is a core-periphery
structure, similar to the one found in open source software production,
and things are easier when the core team people already know each
others: \citet{NemotoGloorLaubacher11} pointed out that ''the more
cohesive and more centralized the collaboration network, and the more
network members were already collaborating before starting to work
together on an article, the faster the article they work on will be
promoted or feature''. In the same time, \citet{ChenRenRiedl10},
evaluating diversity according to an evaluation of the interest of
the persons via their contribution, showed that ''increased diversity
in experience with Wikipedia increases group productivity and decreases
member withdrawal - up to a point. Beyond that point, group productivity
remains high, but members are more likely to withdraw''. Interestingly
for this theory, \citet[p. 22]{Tureketal10} showed, using Polish
Wikipedia data set, that in what they call ''good teams'', the level
of acquaintance is higher than for normal teams (people having discussed
in the talk pages) as is the level of trust (copy-pasting of existing
text when rewriting an article) and of distrust (deletion of text),
which can be seen as the level of creative work (if people delete
more that means that the consensus is reached more slowly, after more
evaluation of the proposals). This is true for FA, but also when articles'
quality is measured by external experts, as in \citeauthor{ArazyNov10}'s
article (2010), who estimated the impact of local inequality and global
inequality on the quality of the article: having a small team, very
committed (strong local inequality), improves the coordination (and
thus indirectly the quality), and having strong global inequality
(people very invested in Wikipedia and peripheral contributors) improves
the quality of the articles (of course, this work may be extended
to a bigger set of article to be confirmed). \citet{GomezKappenKaltenbrunner11}
also showed that ''once a comment on a Wikipedia article has been
originated, it will derive in a collaborative reciprocal chain between
a very reduced group of contributors'', indicating, and contrary
to the contribution to an article, an ''inverse preferential attachment
process'' for the discussions (p. 8). Finally, \citet{XuYilmazZhang08}
can be seen as a summary of these findings: using an agent simulation,
they retrieved these results, showing that more agents improve the
convergence and the quality of the article, especially if they are
more knowledgeable, and vandalism, if increasing the number of updates,
does not stop an article from being improved (it can be seen as test
which allows to question the team and eventually improves its production).

\paragraph{Conclusion.}

The conclusion can be led to \citet{Arazyetal11}, even if they focused
only on a very small subset of articles (96): ''(1) diversity should
be encouraged, as the creative abrasion that is generated when cognitively
diverse members engage in task-related conflict leads to higher-quality
articles''; (we will just add ''up to a point'' here) ''(2) task
conflict should be managed, as conflict notwithstanding its contribution
to creative abrasion can negatively affect group output'' (we will
come back to this point in the next paragraph); and ''(3) groups
should maintain a balance of both administrative- and content-oriented
members, as both contribute to the collaborative process.''

This echoes more general findings about the efficiency of groups.
As shown by \citet{UzziSpiro05} in the case of musical comedies,
and \citet{Uzzi08} in the case of a social network, for a creative
group to be successful, it needs to fine tune the level of newcomers,
for fresh ideas, in an already constituted group (for trust and common
sharing, or ''cohesion'', especially on what a good article is for
Wikipedia, as, according to \citet{ArazyNov10}, the fact to have
people having experience in the contribution in general, or, as we
named them before, boundary spanners, is even more important than
to have people who involve themselves in the production of the article).
Wikipedia seems to be another proof of this principle and it would
be interesting to calculate Wikipedia's ''Q''-level ''bliss point''.

% Transition

This Q-level may depend on the type of article, more specialized,
''narrow focused'', or more general: \citet{KeeganGergleContractor12},
comparing breaking news with historical articles on the commercial
airline disasters, in addition to find the same results as the articles
already cited about the link between quality and number of editors
or length of the article, showed that breaking news articles are more
often chosen by newbies, and that experienced users may avoid this
kind of article.

These studies on the structure help to understand what is needed to
make an article, but give few information on the making, of the life
of this article, and on the interactions needed for this making, which
is the subject of the following paragraph. 