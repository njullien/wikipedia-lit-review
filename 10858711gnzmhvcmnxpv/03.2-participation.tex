
As \citet{PrasarnphanichWagner11} showed, Wikipedia is an aggregation
of contributors with varying levels of resources and interests, verifying
in that aspect too the fact that it follows the critical mass theory
\citep{MarwellOliver93}\footnote{For a formal model of this phenomenon, see \citet{Rahman08}}.
Most of the studies we read, minus one, looked at the motivations
to do positives things participating in Wikipedia. Quite strangely,
we did not find any study on the reasons for leaving Wikipedia, while
they would be interesting to understand how the contribution and the
benefits it brings change, but also, if it is possible, to prevent
these disaffections. Even if, according to the authors, the study
is preliminary, \citet{OrtegaIzquierdo-Cortazar09}, using survival
analysis techniques, showed that the mean time of participation to
the encyclopedia is between 200 and 400 days for the top ten projects
(with a median between 75 and 200 days), quite an important turnover.

This single study not looking at the motivations to do positive things
is by \citet{ShachafHara10} and looks at troll makers' activity.
It shows more social disorder than real motivations. However, these
activities impact on the information available (we will discuss more
this aspect in section 4) and it would be interesting to better understand
how to cope with these behaviors. It relayed on indirect information
about the four trolls they followed, because it was difficult to enter
in contact with them, as they hide their real identity. But this is
a common difficulty of the studies working on Wikipedians' motivation.
It very hard to be accurate in the same time on what people do and
why they do it, as these pieces of information come from different
origins, because there is few internal information on the participants'
skills, sociological background or motivations: \citet{Lametal11}
used users' page gender box and preference setting, for gender studies,
and report a gender information rate of only 6.5\% for editors (in
the English Wikipedia)\footnote{Even if \citet{Ashton11}, in a theoretical work, argues that the
whole editing and contributing activity is the signature, or the ''wikidentity''
\citep[term from ][]{MallanGiardina09} of a person in Wikipedia and
should be studied as so.}. 

So, most of the studies collected external data, via surveys, which
are hard to connect with an IP number or a Wikipedian login, in order
to link them to the internal data about participation. However this
difficulty, the studies available provide with a good understanding
of the characteristics and of the motivations of the participants.

\citet{GlottSchmidtGhosh10} surveyed Wikipedia users (and producers)
and measured their competency by the level of study, and by computer
skill, and their activeness by the time spent on Wikipedia. Another
option for activeness would be to measure the time spent by the participants
(users and providers), but this has not been done as far as we know.
The quality and the representativeness of these declarative data are
hard to verify. However, what these surveys tell us is that contributors
are of higher level of education, mostly male, older in mean that
Wikipedia users, and that mastering basic computer skill matters to
explain contribution\footnote{\citet{CollierBear12} relied on the English version of Glott et al.'s
survey to study the reasons why female Wikipedia users participate
less. Their explanation is that the encyclopedia is a conflicting
environment, and that these users have a lower confidence in their
expertise.}. According to \citet{LiangChenHsu08}, an for the Chinese Wikipedia
administrators they surveyed, having more personal time, weaker social
belongings, or longer Internet surfing time, increase the motivation
for being administrators. When the gap is bridged, the socio-demographic
variables are significantly less explaining of the difference between
contributors: there is, for example, no significant gender difference
in editing between registered Wikipedians \citep[in the English Wikipedia,][]{Antinetal11}. 

In addition to socio-demographic and skills variables, and still using
the survey method, \citet{Amichai-Hamburgeretal08} showed that psychological
characteristics such as agreeableness, openness, or conscientiousness,
are variables to take into account to explain the contribution to
Wikipedia. Focusing only on registered users, \citet{YangLai10} proposed
four types of motivation to explain this involvement: intrinsic (internal
satisfaction such as the pleasure or the fun to contribute, but also
the satisfaction to help by sharing their knowledge, which seems very
important for the most involved participants, according to the results
of a survey amongst Wikipedia administrators by\citealp{BaytiyehPfaffman10}),
extrinsic (image improvement, professional status improvement), external
self-concept-based (recognition by others and especially by peers,
tested by \citet{ZhangZhu11} on the Chinese Wikipedia), and internal
self-concept-based (acting consistently with their vision of themselves).
According to their study, self-concept-based motivations explain the
most the involvement, followed by intrinsic motivations (personal
enjoyment). This is consistent with a precedent study of Wikipedians'
motivation by \citet{Nov07}, which proposed the same methodology
and the same items.

However these global results, an important point is that the motivations
vary over time \citep{ForteBruckman05,Bryantetal05}, and that if,
for the most involved the recognition from the peers ('credit') is
an important motivation (ibid), as is the sense of mission (\citealt{LiangChenHsu08},
basing on a survey of Chinese Wikipedia administrators), for most
of the (small) contributors, the will to fix mistake is the principal
motivation, making these people not strongly committed to the project
\citep[relying on a survey of Japan Wikipedia contributors]{Kamataetal10},
a result \citet{DejeanJullien15} also found for the French Wikipedia
contributors. Using a qualitative methodology (20 semi-guided interviews),
\citet{Antin11} showed the large gap between readers (or occasional
contributors) and regular ones, especially regarding the feeling of
being part of the Wikipedia ''community'' and how this may refrain
from participating. This is explained by the fact that there is a
process of acculturation to Wikipedia: the future contributors are
firstly readers ''dipping their toes in to passively participate
while learning more about a complex system'' (\citealp{AntinCheshire10},
but surveying only US college students), even if the quicker the process
is, the greater the chance people become active contributors are (\citealp{DejeanJullien15},
surveying French Wikipédia's users and contributors, \citealp{PancieraHalfakerTerveen09},
analyzing registered contributors' trajectories). This would means
that the motivations to participate are more individual and internal
and are present since the beginning\footnote{\citet{PrasarnphanichWagner09} defended the idea that ''altruistic''
motivations prevail in Wikipedia, which seems going against this analysis.
But in their study they surveyed 60 very involved Wikipedians, so,
according to what was said, people for who the sense of the community
is the stronger. And the majority of their respondents had mixed motivations.

Regarding the presence of the motivations since the beginning, in
addition to \citet{PancieraHalfakerTerveen09,DejeanJullien15}, already
mentioned, a survey of students from U.S. universities contributing
to the Wikipedia content as part of their course work showed that
''intentions to continue contributing are influenced by the initial
attitude towards the class'' \citep{Zubeetal12}.}. 

In other words, as for open source \citep{LakhaniWolf05,Shah06,Scacchi07}
or professional communities \citep{JullienRoudautleSquin11}, this
may be an illustration of the idea of a path, or ''career'' in the
community \citep[in the sens given by][]{Becker60,Becker63}. To skip
from correcting a mistake to becoming a regular contributor, or an
administrator, would be an additional commitment, which would occur
for reasons developed during the attendance of the project as the
development of this sense of ''community'', i.e. the individual
acceptance of the rules of the organization, as showed by \citet{Pentzold10},
on his study of the meaning of the term community by the very involved
participants of the Wikipedia-1 mailing list (the surveys by \citet{ChoChenChung10}
of 223 English Wikipedians, by \citet{Hoetal11} on the Chinese Wikipedians,
and by \citet{SchroerHertel09} on the German ones all found a link
between this ''sense of belonging'' and the will to contribute).
\citet{KitturPendletonKraut09} also showed how the people modify
their practices of contributing when integrating the Wikiproject,
toward more administrative tasks, according to the group requirement.

This leads to the definition of the activities and the outputs of
this group, or, said differently, the patterns of interaction.