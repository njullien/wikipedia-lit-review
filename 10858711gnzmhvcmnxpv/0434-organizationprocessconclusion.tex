\citet{OttoSimon08} proposed a global model to estimate the evolution
of Wikipedia participants and to evaluate the impact of management
on people's willing to participate, and on the production, which can
be seen as a metaphor of the findings presented in this section. They
used Douglas' grid/group framework \citeyearpar{Douglas70,Douglas78},
where people's behavior is more or less constraint by their commitment
to the group (high/low) and by the structure of the organization (high/low).
Their model shows that ''while an open environment accelerates the
growth of an online network at the early stage, openness may negatively
impact quality and subsequently the attractiveness of the network,
so that users will be less inclined to join or to participate in the
network'' (p. 346). Actually, this seems to be true at each level
of the project (the global language project, the thematic projects
and the articles).

Wikipedia would be a new system controlled by an oligarchy based on
expertise (\citet{HansenBerenteLyytinen09,Hartelius10}, extended
by \citealp{Pfister11}), creating, according to Konieczny \citeyearpar{Konieczny09,Konieczny10},
and his detail discussion of the model of governance in Wikipedia,
a \citeauthor{Mintzberg07}'s model of adhocracy (2007), ''one closely
connected to open-source development models found in the FOSS movement''.
Editors at Wikipedia would ''share the adhocratic values of flat
hierarchy, decentralization, little managerial control, and ad-hoc
creation of informal multidisciplinary teams. Like individuals throughout
most of the FOSS movement, they are highly motivated\textemdash not
by potential financial gain, but by their project\textquoteright s
ideology'' \citep[p. 277]{Konieczny10} (but also, we add, by the
social connexions, and the social and knowledge rewards the participation
provides). However, if, according to him, ''in traditional adhocracies,
individuals are bound by rules that cannot be altered; at Wikipedia,
by contrast, there is no rule that cannot be altered if the community
so desires'', beside the main iron rules (the five pillars), seemingly.
In addition to that, ''in Wikipedia\textquoteright s adhocracy, the
editors not only \textquotedblleft capture opportunities,\textquotedblright{}
but they also can create those opportunities, since editors can change
all policies and so enjoy an unprecedented degree of empowerment''
(something stressed by \citealp{HansenBerenteLyytinen09,Hartelius10,Pfister11}).
More studies should be pursued to see if this model is a new implementation
of the global adhocracy model or if it falls into the model proposed
by \citet{MintzbergMcHugh05}.