%% Evaluation of team members cannot be done at broadly but depends on the specific context.
This article's goal was twofold: to advocate for a more manager-specific approach to evaluating the composition of a good virtual member, especially in the case of FLOSS inspired organizations and to propose a methodology to make this assessment explicit. We do not pretend to have provided any categorization of types of managers on the basis of their preferences. However, our experiment clearly shows that while a set of variables relevant to the evaluation of team members can be identified within a domain, different managers rely on different subsets of factors, and weight the factors differently. In light of prior research showing that competencies and context influence effectiveness within teams, there is a need for a methodology to help managers encode the relative importance of individual attributes in the context of the group.

%% The proposed methodology - which consists of three steps - worked in testing.
Furthermore, we demonstrated that our 3-step methodology can be used to create a model which was accepted by the FLOSS managers we included in our experiment. The three steps consist in defining the variables according to the literature and to validate them with practitioners, selecting an appropriate subset of criteria for each individual manager, and finally eliciting a model of the manager's contributors evaluation through an interactive and iterative process using fictitious contributors profiles.

%% Impact of the research for practitioners and researchers
In proposing this methodology, we contributed towards filling the gap between operational research techniques and information management sciences regarding the expression of different points of view, and the automation of the process, here in the manager/team relationship. We expect that this methodology can provide several benefits to managers and organizations, especially those based on a FLOSS or inner source structure, by enabling appropriate team member selection, improving team member steering, and by increasing the transparency of the manager's evaluation. We are confident that our method can be applied to different situations, either in another domain exhibiting virtual teaming, or in other situations where multi-criteria selection is applied. It is important to mention that the proposed methodology focuses on modeling the preferences of each individual DM and not on the proposal of a general model of how managers in general behave or how they should behave. Such a study would follow as part of a broader experiment in a future contribution.

%% Future work
In order to apply the proposed methodology more broadly, additional work is needed. Streamlining the second and the third phases of the methodology would reduce the risk of inconsistencies due to the current span of the process. This may be done through the construction of a web-based platform as well as through algorithmic improvements, including the exploration of approximative optimization approaches. Furthermore, in order for this approach to be widely adopted, it would be necessary to identify the sets of variables applicable for various other virtual knowledge production domains, as we have currently only explored the case of FLOSS communities. 
%%%% Modifications here %%%
The next steps regarding the use of this methodology for the analysis of virtual teams would be to do the same at group level, and then to study the connexions between team member qualities and group performance, contributing to the discussion opened by \cite{KudaravalliFarajJohnson17} on that matter.

%%% End of modifications %%%%
We hope that this article demonstrates the interest in modeling managers' preferences of (virtual) team member attributes, and also provides an explanation of the benefits this would provide, both scientifically and practically.


