\subsection{Team member characteristics, leadership and team performance}

%% Managing teams is the source of competitive advantage for firms
The competitive advantage of firms has, since the seminal works of \cite{LocanderNapierScamell79,KogutZander92}, been deemed to lie in their capacity for organizing the sharing and transfer of knowledge of -individuals and groups within the organization. More specifically, \cite{Grant96} asserted that the organization does not create knowledge, but selectively selects and coordinates the specialist knowledge embedded within its employees. This is true not only at the scale of a firm, or within the boundaries of a firm, as the goal of the organization becomes to identify, select and coordinate, dynamically, the knowledge capacities required for a project \citep{Teeceal97} through efficient teaming \citep{WuchtyJonesUzzi07}. Thus the objective of the organization is to balance the tacit and explicit knowledge to allow team members to communicate efficiently \citep{Nonaka94} to aggregate individual pieces of knowledge \citep{Grant96}, or, in other words, to manage the transactive memory \citep{Wegner87}. As explained by \cite{TiwanaMcLean05}, following \cite{Grant96}, the ``information systems development team's creativity is predicted by the extent to which its members integrate their specialized expertise to jointly develop project concepts, designs, and solutions.''

%% The attributes required by team members are varied
Because of the broad range of responsibilities and competencies within the IT field, organizational fit must be examined at the level of the individual and the group, rather than generalized to the entire profession \citep{wingreen:2017:professionals}. The context in which competencies are deployed \citep{bassellier:2001:information}, or the ``diverse specialized knowledge in a team, the quality of intra--team working relationships, and  members' cross--domain absorptive capacity---do not engender creativity by themselves; they do so primarily because they enhance integration of individual knowledge at the project level'' \citep{TiwanaMcLean05}. 

%% The role of the manager is to align people with the organization
The understanding of the role of the manager has evolved from principal--agent control theories to stewardship models \citep{DavisSchoormanDonaldson97}, where the managers, or principals, have motives and motivations aligned with their agents, or team. The alignment is based on the construction of a covenant relationship between individuals and the organization: they commit to work toward the same goal, without taking advantage of one another, for the benefit of all parties \citep{Hernandez12}.  \cite{Wageman01} found that while design choices and hands--on coaching both impact the well--being of self--managing teams, only leaders' design activities affect task performance: ``design and coaching interact, well--designed teams are helped more by effective coaching.'' In modern dynamically-composed virtual teams, such as software teams, goals and tasks are no longer stable and the selection of members does not depend strictly on a candidate's ability to perform a specific role, yet individual competencies must still be aligned with team requirements \citep{tannenbaum:2012:teams}.

%% The role of the manager is to assemble and manage a team
In this view of the firm of managing knowledge via ad-hoc, increasingly virtual teams, the project manager is the ``chief executive'' of the ``temporary organization'' \citep{TurnerMuller03} responsible not only for motivating team members---which can be especially challenging in a global, distributed environment \citep{beecham:2014:motivating,noll:2016:global}---but also selecting them, as team assembly mechanisms can determine team performance \citep{Guimeraetal05}, especially in creative teams such as those engaged in building knowledge commons \citep{HessOstrom06b}. Both the quality of the team and good management are motivation factors in virtual software development teams \citep{beecham2015motivates}, while poor feedback, inequity and unfair rewards are demotivating and of particular concern in this environment \citep{deshpande:2009:management}. Consequently, the best way to improve team efficiency may be to provide the leaders of (virtual) teams with a methodology to systematically and transparently describe the characteristics which matter to them in the evaluation and selection and development of team members.

%% What we are looking at and what we are not looking at
The precise elicitation of key characteristics for team members, from the perspective of the manager, is at the core of what we propose in this article, rather than other factors contributing to efficiency. %% TAMARA
In this we follow the shift in virtual team research from skills and abilities of people working in virtual teams to team composition \citep{Gilsonetal15}, specifically the skills and attributes a specific manager values in the composition of a particular team. 
In contrast to \cite{Kozlowskietal15}, we consider the characteristics of the team members, rather than the acknowledged impact of the information system design. Due to our interest in applications such as team composition, which is typically the purview of the manager, our method strives to elicit one perspective---the manager's---rather than incorporating impressions from multiple team members as in \cite{sarker:2011:role}'s approach involving identifying the extent to which a team member is valued by her or his peers. While we examine the point of view of the manager, we are not, like \cite{Dionneetal04}, looking at the impact of leadership on processes. 
% Team composition has been demonstrated to affect efficiency by impacting the \cite{vanKnippenbergDeDreuHoman04} process. \cite{Guimeraetal05} proposed analyzing the links between group diversity, process and performance and showed that team assembly mechanisms determine team performance for creative teams, especially in teams building knowledge commons \citep{HessOstrom06b}. 
It is important to note that we are not looking for criteria with the greatest average impact, but for a fine elicitation of each manager's evaluation model of virtual team member attributes. 
The criteria are therefore subjective, with a rough granularity; for instance a person could be viewed as `good' or `bad' at communication. 

%% These are the characteristics we propose to look at
Characteristics of team members can be categorized into technical ``multifunctional knowledge'' and the ``teamwork capability of team members'' such as communication skills which combine with other factors---such as the ``working relationship'' between teammates---to bring knowledge together \citep{ChenLin04}. \cite{bassellier:2004:business} uses the categories ``business competence'' and ``IT competence'' to describe, respectively, organization-specific knowledge and more generic, transferable skills. We opted to use the classification system which distinguishes between universal human qualities, namely psychological factors, and domain-specific technical skills.
It has been demonstrated that both psychological components and technical skills are
relevant in virtual teams \citep{luse:2013:personality}, and not only technical but also social factors can be observed through outputs in a virtual context \citep{dabbish:2012:social}.
By using a psychological/technical division our solution can be applied in different
contexts involving virtual teams, through customization of the technical component to fit the specific domain.
  
%% How psychological characteristics are operationalized
\cite{ChenLin04} proposed five descriptors of team member characteristics based on \cite{Myersetal85}'s psychological indicators. However, these are coarse-grained, and incorporate some measure of technical skills with descriptors such as ``functional expertise.'' Consequently, we relied on the personality indicator method described by \cite{Driskelletal06}. This work examined the relationship between team member personality traits and team effectiveness, basing the traits on the ten personality indicators (BFI-10) \citep{McCraeCosta89}. The five indicators which describe the relationship between team members and team effectiveness are: {\it communication}, {\it commitment}, {\it collaboration}, {\it handling of pressure}, and {\it creativity} \citep{Driskelletal06}. Measuring psychometric characteristics using the abbreviated BFI-10 scale normally requires a number of questions. However, \cite{rammstedt:2007:measuring} showed that a ten-question instrument provides a robust measure of the traits. The questions actually represent five indicators (the big five) in both negative and positive incarnations. It can be reduced when time is a constraint and the research does not rely on a precise measure \citep{GoslingRentfrowSwann03}, as in our case. These five indicators provide a solid approach to the analysis of the psycho-social dimensions of teaming. Technical abilities are more variable and depend on domain and will therefore be detailed later, as part of our method.

%% Ineffectiveness of current techniques of assessment
In a case where the criteria are known and are the same for every unit of analysis, classical data analysis methodologies such as linear regression or decision tree modeling can be used. The reader who is interested in a review of these findings may consult \cite{Stewart06,Driskelletal06}. However, these results do not deliver actionable managerial recommendations. What is lacking is a technique to assess individuals in the context of their relations with the manager, the relation which has a significant impact on team performance. In other words, to improve the efficiency of teaming, we propose to develop a non arbitrary, non ad-hoc evaluation scale based on multiple criteria, which encapsulates the perspective of the manager. When multiple, potentially conflicting evaluation criteria must be taken into account, and the individual preferences of the manager should be captured, it is natural to turn towards Multi-Criteria Decision Aiding (MCDA) techniques.
% We need more here, stressing more on what is new in our methods, what we cannot, for instance use Monte-Carlo methods, such as \cite{Stewart93}. Qualitativity, impreciseness are of outmost importance to justify this new methodology and the breakthrough it provides

\subsection{Multi-Criteria Decision Aiding}\label{sec:MCDA}

Multi-Criteria Decision Aiding is the study of decision problems, methods and tools which may be used in order to assist a decision maker (DM) in reaching a decision when faced with a set of alternatives, described via multiple---often conflicting---properties or criteria. In the context of this paper, the DM is the team manager, and the alternatives are the team members, while the evaluation criteria are the attributes used to evaluate the team members.

Usually, three types of decision problems are found in the MCDA context \citep{roy96}. First is the {\em choice} problem, which aims to recommend a subset of alternatives, as restricted as possible, which contain the ``satisfactory'' outcomes. Next is the {\em sorting} problem, which attempts to assign each alternative into predefined categories or classes, or to construct a qualitative evaluation scale. The last is the {\em ranking} problem, which aims to order the alternatives by decreasing order of preferences, or to build a quantitative evaluation scale.


%%%%% Modifications here %%%%%%

Various methodologies and preference models have been proposed to support DMs facing a multi-criteria decision problem \citep{bouyssouMarchantPirlotTsoukiasVincke06,keeneyRaiffa76,roy96}. {\em Outranking methods} compare any two alternatives, based on the preferences of the DM on the set of criteria, using a majority rule. Alternately, methods based on {\em multi-attribute value theory} aim to construct a numerical representation of the DM's preference on the set of alternatives $A$. The main difference between these two methodological schools lies in the way in which the alternatives are compared and the type of information required from the DM. Outranking methods are preferred if the evaluations of the alternatives are primarily qualitative, if the DM would like to include a measure of imprecision about personal preferences in the model, and when a human-readable evaluation model is desired. Value-based methods can be favored if compensatory behavior of the DM should be modeled, and when the evaluation of the alternatives should be summarized by a single value (as in the case of accounting, for instance).

%%% End of modif %%%%


Our aim, in this article, is to evaluate team members by building a transparent and understandable evaluation model which integrates the manager's preferences. The evaluation of team members relies on qualitative criteria. Furthermore, we require that the overall evaluation also delivers a qualitative scale. Last but not least, in order to develop a functional tool for teams, we require a readable evaluation model which enables the manager to understand the strengths and weaknesses of the team members, while providing details on improvement. All of these arguments are in favor of outranking preference models, more specifically an outranking sorting technique.

Among the possible outranking sorting algorithms, we chose Majority-Rule Sorting (MR-Sort), which is a simplified version of the classic ElectreTRI~\citep{FMR05,mousseauuseroriented2000,royoutranking1991}. Our version is similar to the one axiomatized in \citep{bouyssouaxiomatic2007-1,bouyssouaxiomatic2007-2}. This technique allows us to build an overall qualitative evaluation scale for the evaluation of the contributors, while presenting a very readable and operational model. For a detailed mathematical description of the approach, please refer to the Appendix. % \ref{app:mrsort}

This MR-Sort model can be extended in different ways to model more accurately the preferential behavior of the DM. For example, even when the coalition is strong enough, a criterion may veto the outranking situation. This leads to the MRV-Sort model, in which an alternative $a$ is said to outrank a frontier $b_{h-1}$ if and only if there is a sufficient coalition of criteria supporting the assertion ``$a$ is at least as good as $b_{h-1}$'' and no criterion strongly opposes (vetoes) that assertion. An alternative $a$ is therefore in a veto relation (denoted with $\operatorname{V}$) with a profile $b_{h-1}$ when:
\begin{equation}
	a\operatorname{V}b_{h-1} \iff \exists j\in J: g_j(a) < g_j(b^{v}_{h-1}).\label{eq:binaryveto}
\end{equation}
The veto profile $b^{v}_{h-1}$ represents the minimum level of performance that an alternative needs to have in order to be allowed into category $c_h$ via the weighted coalition of criteria in favor of this assignment. If on any criterion an alternative has a lower performance than the veto profile of $c_h$, then it is forbidden to be assigned to $c_h$ or above. 

Other extensions to the classical MR-Sort models exist. For instance, \cite{Meyer2017216} extended the MRV-Sort approach in order to handle large positive performances (resulting in what is called dictator effects). This work shows that various options of combining the veto and dictator effects lead to different preference models. 

These extensions lead to more and more flexible and complex models as their number of parameters increases. Consequently, they allow the modeling of increasingly complex preferential statements of the DM. 

Several techniques have been proposed in the literature to learn the parameters of outranking-based multi-criteria sorting models as an alternative to directly asking the DM to provide them. \cite{mousseauusing2001,mousseauinferringelectre1998,theusing2002} propose finding the model parameters through the use of {\em assignment examples}. More specifically, the DM is asked in a first step to  assign a few {\em well known} alternatives to the predefined categories. Then, from these assignment examples, the model parameters are extracted using linear, mixed integer linear or non-linear programs. Some of these approaches may not be successfully used on large sets of assignment examples; therefore, more recently, approximative approaches have been proposed by \cite{olteanumeyer2014,sobriemousseaupirlot2013}. Finally, \cite{Meyer2017216} proposed mixed integer programming techniques to learn the parameters of extensions of the MR-Sort method. 



\subsection{A specific case: the FLOSS team}

%%% Modif here %%%
Before proceeding to the main contribution of this work, we offer an explanation as to why we used the case of Free/Libre/Open Source Software (FLOSS) software development teams to illustrate the application of the method. There were three reasons for concluding that FLOSS provides a generalizable example of the interest of using such a technique to evaluate virtual teams.
%%% End of modif %%%%


%% FLOSS is a suitable example because the teaming model is similar to other sorts of software development and virtual knowledge production.
Firstly, and as said before, FLOSS is considered a model for firms employing virtual teaming for the production of knowledge (in this case, software), especially in the IS literature \citep{Fitzgerald06}. The fact that social skills in conjunction with leadership behavior affect team motivation and performance has already been described in general and are highly applicable to the context of software development. \cite{Moreno-Leonetal16} claimed that programming skills develop better when social skills are present, while \cite{beecham2015motivates} found that soft skills such as cultural sensitivity and appreciation for the individual are important to members of globally distributed teams. Computational thinking, or the ability to solve problems, design systems, and draw on concepts fundamental to computer science, is said to be positively correlated with three personality factors: openness, extraversion and conscientiousness \citep{gonzales:2016:does}. FLOSS research has likewise confirmed the importance of contributors possessing both technical and social skills \citep{david:2007:free,zhou:2012:what,carillo:2014:only,steinmacher:2015:systematic,wei:2016:roles,poo:2016:herding,carillo:2017:what}. 

%% FLOSS is an appropriate example because there is evidence that different projects have different priorities/needs with regard to skills.

%%%% Modif here %%%%

Secondly, there are reasons to believe that the FLOSS example will show various managerial preferences at the team level, rather than presenting a uniform set of requirements across all projects. As our claim is that our technique helps detecting those differences, it was important to choose a domain where there was something to detect. 
%%%% End of modif %%%%%
Recent literature on FLOSS teaming suggests that there is some disagreement on what constitutes a good team member. While it is widely agreed that social and technical capabilities are required for software development, the ideal balance of qualities varies by group, even when projects of a similar type are compared \citep{carillo:2016:dose}. \cite{kalliamvakou:2009:measuring} devised a model which looks at different potential contributions and calculates the positive or negative effect and the weight of each type of action. The weights can be set at the project level to reflect individual views of the importance of actions to the project. They found that in the GNOME Desktop, the developer with the highest positive contribution had the fewest negative actions, whereas in GNOME-VFS, the highest ranked participants also had the most negative contributions, suggesting that these two groups within the GNOME community have different tolerances for negative actions. The same difference in perceptions of social skills can be seen in public statements by team leaders. For instance, Dreamwidth prides itself on its inclusivity \footnote{Denise Paolucci and Mark Smith, Web 2.0 Expo SF, 2010, \url{https://www.slideshare.net/dreamwidth/build-your-own-contributors-one-part-at-a-time}.}, while Linus Torvalds, the founder of Linux, has famously said, ``Talk is cheap. Show me the code,''\footnote{Linux kernel mailing list. Google Groups, \url{https://groups.google.com/d/msg/fa.linux.kernel/iQtWFALi4JA/eSzv64_tOvoJ}. Other quotations from L. Torvalds may be found in Wikiquote: \url{https://en.wikiquote.org/wiki/Linus_Torvalds}.} suggesting that technical capability/proposals are highly valued in his community over other attributes. 

%% Quite aside from its value as an example, the FLOSS case demonstrates a clear need for an assessment process, as current methods are insufficient.
Finally, existing processes for evaluating team members in FLOSS groups have clear limitations. Most widespread is the idea of meritocracy, where the best developer is easily recognized by the quality of contribution, and gains the greatest recognition through evaluation by peers \citep{scacchi:2007:free}. However, meritocratic cultures have been demonstrated to deliver biased observations \citep{castilla:2010:paradox}, and FLOSS communities have been specifically criticized for this shortcoming \citep{reagle:2012:free,nafus:2012:patches}. There is evidence that FLOSS communities incorporate social structures which affect how contributions are received \citep{vonkrogh:2003:community,ducheneaut:2005:socialization,tsay:2014:influence,keertipati:2016:exploring}. FLOSS community managers wrestling with the difficulty of identifying ``poisonous'' behaviors \citep{carillo:2016:dose}, or facing decisions about the allocation of voting rights---a scenario posed by one of our participants---can benefit from a methodology which aids in articulating the ways in which a person succeeds or fails to perform within the team. As membership in FLOSS projects is based on self-selection, there is potentially greater variety in team members' profiles than in a classic company, where the most extreme profiles would have been rejected in the hiring process.

All these reasons, in addition to our access to FLOSS community managers, explain why we chose to illustrate our methodology with this example. Consequently, for the rest of the paper, FLOSS contributors will be described as team members, and community managers as decision makers (DM).

The methodology in its generality will be presented in the next section, illustrated with the case of FLOSS when this helps to better understand it, without losing in generality. Section~\ref{sec:results} will detail the FLOSS case, as an illustration of what this methodology can provide to the practitioner and the researcher.
