The most obvious presentation of the product Wikipedia is by its number
of articles, or the total number of pages (including redirection and
discussion), available on the welcome page, and its size. The coverage
is also available as internal data\footnote{\url{http://en.Wikipedia.org/wiki/Portal:Contents/Indexes}},
which reveals the structure of the encyclopedia, but also what is
one of the basic rules of the encyclopedia: to publish verifiable
and not research information.

On the coverage, the study by \citet{HalavaisLackaff08} argues that
''Wikipedia\textquoteright s worth lies not only with accuracy, but
also in its breadth of subject coverage''. They employed two methods
to examine the subject coverage of Wikipedia. The first was to compare
a sample of Wikipedia's \textquotedblleft topical scope and coverage\textquotedblright{}
with a similar sample from Bowker\textquoteright s Books in Print.
Generally, they determined that Wikipedia\textquoteright s enormous
size means that \textquotedblleft even in the least covered areas
{[}. . .{]} Wikipedia does well\textquotedblright . The second aspect
of their study compared a topical study of Wikipedia with various
scholarly print sources such as Encyclopedia of Linguistics, New Princeton
Encyclopedia of Poetry and Poetics, and Encyclopedia of Physics. They
determined that Wikipedia\textquoteright s specialized research coverage
is more limited than that of these specialized print sources. They
explained that by the organic nature of Wikipedia: driven by contributor
interest and engagement, some topics develop quickly (popular culture
and physical science) whereas others increase more slowly. This surmise
is validated by \citet{Denningetal05}, and \citet{RoyalKapila09},
who showed that ''Some topics are covered more comprehensively than
others, and the predictors of these biases include recency, importance,
population, and financial wealth'', and coherent with the analyses
done by \citet{Ibaetal10,KeeganGergleContractor12} (see also the
discussion in \ref{sec:The-pieces-of-knowledge}).